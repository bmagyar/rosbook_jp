% !TEX root = ./rosbook_jp.tex
%-------------------------------------------------------------------------------
\chapterimage{chapter_head_12.pdf}
%-------------------------------------------------------------------------------
\chapter*{はじめに}

%-------------------------------------------------------------------------------
\section*{ROSの解説書}

本書はROS(Robot Operating System)の解説書です。ROSを全く知らない方も、本書に従えば、容易にROSの導入や基本的な概念の習得、SLAMなどが体験できます。さらに既にROSを使ったことがある方も、Gazebo、MoveItの使用法など、より高度な知識を得ることができます。ロボットのプログラミングは通常、高度で専門的な知識が必要ですが、ROSでは世界中の研究者や開発者が開発した多くの
便利なプログラムが提供されており、これを使えば短期間で専門的なプログラムを作成し、世界に向けて発信できます。是非、あなた自身のプログラムを作成、公開し、ROSコミュニティの一員として、ロボティックスの発展に貢献してください。

%-------------------------------------------------------------------------------
\section*{ライセンス}

本書はクリエイティブ・コモンズ・表示 - 非営利 4.0 国際・ライセンスで提供されています。詳しくは、 \url{http://creativecommons.org/licenses/by-nc/4.0/deed.ja} を参照して下さい。

%-------------------------------------------------------------------------------
\section*{引用について}

本書の一部はOpen Source Robotics Foundation, Inc. (OSRF)が管理しているウィキサイト(http://wiki.ros.org/)を参照しています。このサイトは クリエイティブ・コモンズ 表示 3.0 非移植 ライセンスの下に提供されています。詳しくは、http://creativecommons.org/licenses/by/3.0/ を参照してください。他の引用については本文中で表示しています。

%-------------------------------------------------------------------------------
\section*{進化する本}

本書の内容や図も以下のGithubリポジトリですべて公開しています。本書の内容で間違いなどがございましたら,以下のリポジトリから、IssueやPull Requestにてお知らせください。

\begin{lstlisting}
https://github.com/irvs/rosbook_jp
\end{lstlisting}

\noindent 本書で使用したプログラムも以下のGithubリポジトリですべて公開しています。間違いやご意見などは,以下のリポジトリから、IssueやPull Requestにてお知らせください。

\begin{lstlisting}
https://github.com/irvs/irvs_ros_tutorials
https://github.com/irvs/rosbook_kobuki
https://github.com/irvs/rosbook_robot_arm
\end{lstlisting}

%-------------------------------------------------------------------------------
\section*{本のバージョン}

本書の最新バージョンは以下のアドレスからダウンロードできます。本書のバージョンは、1ページ目の下に記載されています。

\begin{lstlisting}
http://irvs.github.io/rosbook_jp/
\end{lstlisting}

%-------------------------------------------------------------------------------
\section*{日本ROSユーザーコミュニティー}

日本では、東京オープンソースロボティクス協会(TORK)やROS JAPAN Users Group が中心になってセミナーや勉強会を開催し、ROSの普及に努めています。このユーザーコミュニティーも積極的にご利用ください。

\begin{lstlisting}
TORK     http://opensource-robotics.tokyo.jp/
ROS JAPAN Users Group     https://groups.google.com/forum/#!forum/ros-japan-users/
\end{lstlisting}

%-------------------------------------------------------------------------------
